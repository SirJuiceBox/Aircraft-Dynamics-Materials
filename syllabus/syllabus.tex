\documentclass[9pt]{article}

\usepackage{fullpage}
\usepackage{hyperref}
\usepackage{enumitem}
\usepackage{multicol}
% \usepackage[normalem]{ulem}

\addtolength{\topmargin}{-.25in}
\addtolength{\textheight}{0.5in}
% \setlength{\parindent}{0pt}
\setlength{\multicolsep}{2pt}

\title{Syllabus: ASEN 3728 Aircraft Dynamics}
\author{Professor Zachary Sunberg}
\date{Spring 2024}

\begin{document}

\maketitle

\section*{Meetings}

T/TH 1:00-2:15 pm, AERO 120.

\section*{Course Staff}

\subsection*{Instructor}
Professor Zachary Sunberg\\
AERO 263 \href{mailto://zachary.sunberg@colorado.edu}{\nolinkurl{zachary.sunberg@colorado.edu}} (before emailing, please see if there are instructions on canvas to contact course staff in other ways, such as posting on a discussion board)\\
\textbf{Office Hours}: To be Posted on Canvas

\subsection*{Teaching Assistants}

\textbf{Office Hours}: To be Posted on Canvas

\begin{itemize}[noitemsep]
    \item Kelvin Aladum (\href{mailto://Kelvin.Aladum@colorado.edu}{Kelvin.Aladum@colorado.edu})
    \item Aidan Luczkow (\href{mailto://ailu9881@colorado.edu}{ailu9881@colorado.edu})
    \item Shaya Naimi (\href{mailto://Shaya.Naimi@colorado.edu}{Shaya.Naimi@colorado.edu})
    \item John Dallin (\href{mailto://john.dallin@colorado.edu}{John.Dallin@colorado.edu})
    \item Connor Larson (\href{mailto://Connor.Larson-2@colorado.edu}{Connor.Larson-2@colorado.edu})
\end{itemize}

\section*{Textbook}

\textbf{Bernard Etkin and Lloyd Reid, \textit{Dynamics of Flight: Stability and Control, 3rd Edition}}. 1996, John Wiley and Sons.

\section*{Prerequisites}

ASEN 2702, 2703, 2704, and APPM 2360 (min grade C-).

\section*{Overview}

This course covers the key ideas that enable: (i) an understanding of
how aircraft work and tools for quantitative analysis, and (ii) design
methods to achieve specified dynamical behavior. Because aircraft exist
in many different forms, and new designs continue to be developed, the
focus is on the common principles that underlie atmospheric flight, so
that a solid basis can be formed for future work in any direction.
Concrete treatment of these ideas, tools, and methods is provided
through working problems consisting
of analysis, simulation, and design, including development of
simulation models for two very different vehicles: a quad-copter and a
conventional airplane.

In their full expression, aircraft dynamics possess astounding
complexity. It is a tribute to the ideas developed by aviation's
pioneers that a relatively simple understanding can often be obtained,
leading to clear insights and design principles. While these concepts
are not inherently difficult, they do lie outside most common
experience, and they depend on new nomenclature and strange notation
that can seem overwhelming at first. It is only through diligent and
careful use of this new language that the underlying simplicity can be
grasped and conveyed on exams; mastery of the language of aircraft
dynamics is perhaps the most important predictor for success in the
course.

The course has been designed to develop a conceptual grasp of the key
ideas below, and to demonstrate proficiency in using these concepts to
solve problems, construct and validate simulations, and to explain
behaviors and results obtained. In particular, engineering reasoning
skills using these concepts are stressed in assignment solutions and
examinations. The key learning objectives are:\\

\begin{itemize}[nosep]
\item Vector mechanics
  \begin{itemize}[nosep]
  \item Vector representation in coordinate frames
  \item Change of coordinate frame representation (coordinate rotation)
  \item Relative motion, frame derivatives
  \item Change of derivative frame: velocity rule
  \end{itemize}
\item How aircraft dynamics models are created and what the terms mean
  \begin{itemize}[nosep]
  \item 3D rigid body translational model
    \begin{itemize}[nosep]
    \item Kinematics
    \item Dynamics, external forces
    \item Effects of wind
    \end{itemize}
  \item 3D rigid body rotational model
      \begin{itemize}[nosep]
    \item Kinematics, Euler angle attitude representation
    \item Dynamics, Euler moment equations, external moments
    \end{itemize}
  \item External forces and moments
      \begin{itemize}[nosep]
    \item Aerodynamic effects
    \item Control effects
    \item Steady flight conditions, trim states
    \end{itemize}
  \end{itemize}
\item How aircraft dynamics models are simulated
    \begin{itemize}[nosep]
    \item State space models
    \item Numerical integration
  \end{itemize}
\item How dynamical behavior is understood and specified
    \begin{itemize}[nosep]
    \item Linearization
    \item Decoupling
    \item Stability derivatives
    \item Modal solutions
    \item Stability characterizations
    \item Modal specifications
  \end{itemize}
\item How feedback control is designed to meet behavioral objectives
    \begin{itemize}[nosep]
    \item Sensor/feedback selection, control structure and gain selection
    \item Effects on mode eigenvalues
  \end{itemize}
\end{itemize}

\section*{Course Components}\label{course-components}

Material and concepts are introduced, and student mastery is evaluated
using several mechanisms throughout the course:

\textbf{Reading} -- The textbook provides the essential basis for the
course, including the concepts, terminology, notation, methods, and
examples used to convey the course topics. Specific reading assignments
will be given covering key sections of the book; some book sections are
not covered in the course. Some supplementary material will also be
provided. {The textbook contains a wealth of information, but the
concepts and notation are new to most; some sections need to be read
more than once to fully grasp the material}\emph{.}

\textbf{Lectures} -- These are intended to emphasize key ideas and
methods that make the material easier to grasp. They are therefore a
counterpart to the reading, not a replacement. {The value of lectures is
dependent on your participation in them}. Passive ``watching'' will
provide little benefit. {Active note taking is critical to developing
first-hand familiarity with the notation, terminology, and methods, and
to gaining comfort in using them}. Although lectures will be recorded,
this is a poor substitute for your own lecture notes. Questions are
encouraged during lectures and will be prompted often.

\textbf{Homework} -- In the instructor's opinion, homework is the most important tool for learning in this class because it provides essential individual practice in preparation for the exams and experience solving problems that will be faced in the student's career. Homework will consist of
solving problems of varying difficulty and sometimes will also involve
computing. Collaboration on homework is encouraged, but the work you submit must be your own. Students are encouraged to use homework as a means to ensure
their individual mastery of the subject.

\textbf{In-Class and Reading Quizzes} -- These will cover the reading material, and
lectures.  They will usually consist of true-false and multiple-choice-style questions.

\textbf{Exams} -- These are the primary means of evaluation of your
individual grasp of the course material. Exams will include both conceptual questions and
quantitative problems. Precise use of terminology and notation is
stressed. The final exam is comprehensive in that it will contain
material from the entire course, but emphasis will be placed on the
final portion of the course material. \textbf{There will
be a statute of limitations on when exam grades can be corrected. Any
corrections on exam scores must be made before the next exam, or two
weeks after the exam was returned, whichever comes later}.

\section*{Websites}

\begin{itemize}[nosep]
    \item \textbf{Canvas} will be the hub for the course. Links to all other materials can be found there.
\end{itemize}

\section*{Attendance and Participation}

Learning is a collaborative effort between the instructor and students. Students are expected to attend all lectures, ask questions, and participate in discussions. The course staff will encourage attendance through in-class quizzes. \textbf{If a student needs to miss class occasionally, please do NOT notify the course staff.} Several of the lowest in-class quiz scores will be dropped to accommodate absences (see grading breakdown below).

\section*{Grading Philosophy}

Grades are assigned according to an absolute standard designed to
indicate your level of competence in the course material. The final
grade indicates your readiness to continue to the next level in the
curriculum. {The AES faculty have set these standards based on our
education, experience, interactions with industry, government
laboratories, others in academy, and according to the criteria
established by the ABET accreditation board}.

The course grade is primarily dependent on individual measures of
competency, i.e., exams. The other course assignments are designed to
enrich the learning experience and to enhance individual performance,
not to substitute for sub-standard individual competency. This policy
makes it important to use the assignments to enhance your learning.

Grades for the course are earned based on the following criteria:
\begin{quote}
A, A- Demonstrates mastery of the course material in both conceptual and
quantitative aspects.

B+, B Demonstrates comprehensive understanding of the material, with a
solid conceptual grasp of key concepts and strong quantitative work.

B-, C+ Demonstrates good understanding of most key concepts, with few
major quantitative errors.

C Demonstrates satisfying understanding of the material with sufficient
quantitative work.

C- Demonstrates adequate understanding of the material to proceed to the
next level; sufficient quantitative work.

D Very little understanding is evident, consistently poor quantitative
work.

F Unsatisfactory performance.
\end{quote}

Graders will assess whether responses provided by
students reflect knowledge, understanding and reasoning processes that
{meaningfully contribute} to answering questions posed on assignments.
Vacuous responses, e.g., repeating questions,
listing buzzwords, irrelevant diagram drawing, etc., will not suffice. This subject is difficult and
non-intuitive, and since this is the first time most (if not all)
students have seen this material, it is naturally assumed that all
students must work hard and put in effort to learn the concepts.
Therefore, hard work is necessary, but not sufficient by itself, to do
well. Your effort must translate to demonstrable individual
understanding for success.

\subsection*{Grade Breakdown}

\begin{itemize}[noitemsep]
    \item \textbf{10\% Quizzes: In-class and Online.} (lowest 20\% dropped)
    \item \textbf{16\% Homework.} (lowest dropped)
    \item \textbf{44\% Two Midterm Exams.} (22\% each)
    \item \textbf{30\% Final Exam.}
\end{itemize}

\subsection*{Late Policy}

To ensure proper progression through the course, students are expected to begin assignments early and submit homework assignments on time. However, in order to provide for minor unforeseen events or responsibilities, students may turn in late homework assignments within 72 hours of the due date with a 10\% penalty. No homework will be accepted after 72 hours beyond the due date. The lowest homework score will be dropped to account for missed assignments.

\section*{Additional Policies}

{\small
    \subsection*{Classroom Behavior}

Students and faculty are responsible for maintaining an appropriate
learning environment in all instructional settings, whether in person,
remote, or online. Failure to adhere to such behavioral standards may be
subject to discipline. Professional courtesy and sensitivity are
especially important with respect to individuals and topics dealing with
race, color, national origin, sex, pregnancy, age, disability, creed,
religion, sexual orientation, gender identity, gender expression,
veteran status, political affiliation, or political philosophy.\\
~\\
For more information, see the
\href{http://www.colorado.edu/policies/student-classroom-and-course-related-behavior}{{classroom
behavior policy}}, the
\href{https://www.colorado.edu/sccr/student-conduct}{{Student Code of
Conduct}}, and the \href{https://www.colorado.edu/oiec/}{{Office of
Institutional Equity and Compliance}}.

\subsection*{Requirements for Infectious Disease}

Members of the CU Boulder community and visitors to campus must follow
university, department, and building health and safety requirements and
all applicable campus policies and public health guidelines to reduce
the risk of spreading infectious diseases. If public health conditions
require, the university may also invoke related requirements for student
conduct and disability accommodation that will apply to this class.

If you feel ill and think you might have COVID-19 or if you have tested
positive for COVID-19, please stay home and follow the
\href{https://www.cdc.gov/coronavirus/2019-ncov/your-health/isolation.html}{{guidance
of the Centers for Disease Control and Prevention (CDC) for isolation
and testing}}. If you have been in close contact with someone who has
COVID-19 but do not have any symptoms and have not tested positive for
COVID-19, you do not need to stay home but should follow the
\href{https://www.cdc.gov/coronavirus/2019-ncov/your-health/if-you-were-exposed.html}{{guidance
of the CDC for masking and testing}}.

\subsection*{Accommodation for Disabilities, Temporary Medical Conditions, and
Medical Isolation}

If you qualify for accommodations because of a disability, please submit
your accommodation letter from Disability Services to your faculty
member in a timely manner so that your needs can be addressed.~
Disability Services determines accommodations based on documented
disabilities in the academic environment.~ Information on requesting
accommodations is located on
the~\href{https://www.colorado.edu/disabilityservices/}{{Disability
Services website}}. Contact Disability Services at 303-492-8671
or~\href{mailto:dsinfo@colorado.edu}{{dsinfo@colorado.edu}}~ for further
assistance.~ If you have a temporary medical condition,
see~\href{https://www.colorado.edu/disabilityservices/students/temporary-medical-conditions}{{Temporary
Medical Conditions}}~on the Disability Services website.

Students are expected to start on assignments early so that minor temporary medical conditions do not prevent them from turning assignments in on time. In addition, the late policy is designed to accommodate minor temporary medical conditions. If you have a major medical emergency that prevents you from completing an assignment, please contact the instructor as soon as possible to discuss accommodations.

\subsection*{Preferred Student Names and Pronouns}

CU Boulder recognizes that students' legal information doesn't always
align with how they identify. Students may update their preferred names
and pronouns via the student portal; those preferred names and pronouns
are listed on instructors' class rosters. In the absence of such
updates, the name that appears on the class roster is the student's
legal name.

\subsection*{Honor Code}

All students enrolled in a University of Colorado Boulder course are
responsible for knowing and adhering to the
\href{https://www.colorado.edu/sccr/honor-code}{{Honor Code}}.
Violations of the Honor Code may include but are not limited to:
plagiarism (including use of paper writing services or technology
{[}such as essay bots{]}), cheating, fabrication, lying, bribery,
threat, unauthorized access to academic materials, clicker fraud,
submitting the same or similar work in more than one course without
permission from all course instructors involved, and aiding academic
dishonesty.

All incidents of academic misconduct will be reported to Student Conduct
\& Conflict Resolution:
\href{mailto:honor@colorado.edu}{{honor@colorado.edu}}, 303-492-5550.
Students found responsible for violating the
\href{https://www.colorado.edu/sccr/honor-code}{{Honor Code}} will be
assigned resolution outcomes from the Student Conduct \& Conflict
Resolution as well as be subject to academic sanctions from the faculty
member. Visit \href{https://www.colorado.edu/sccr/honor-code}{{Honor
Code}} for more information on the academic integrity policy.

\subsection*{Sexual Misconduct, Discrimination, Harassment and/or Related Retaliation}

CU Boulder is committed to fostering an inclusive and welcoming
learning, working, and living environment. University policy prohibits
\href{https://www.colorado.edu/oiec/policies/discrimination-harassment-policy/protected-class-definitions}{{protected-class}}
discrimination and harassment, sexual misconduct (harassment,
exploitation, and assault), intimate partner violence (dating or
domestic violence), stalking, and related retaliation by or against
members of our community on- and off-campus. These behaviors harm
individuals and our community. The Office of Institutional Equity and
Compliance (OIEC) addresses these concerns, and individuals who have
been subjected to misconduct can contact OIEC at 303-492-2127 or email
\href{mailto:cureport@colorado.edu}{{cureport@colorado.edu}}.
Information about university policies,
\href{https://www.colorado.edu/oiec/reporting-resolutions/making-report}{{reporting
options}}, and
\href{https://www.colorado.edu/oiec/support-resources}{{support
resources}} can be found on the
\href{http://www.colorado.edu/institutionalequity/}{{OIEC website}}.

Please know that faculty and graduate instructors must inform OIEC when
they are made aware of incidents related to these policies regardless of
when or where something occurred. This is to ensure that individuals
impacted receive outreach from OIEC about resolution options and support
resources. To learn more about reporting and support for a variety of
concerns, visit the {\href{https://www.colorado.edu/dontignoreit/}{
Don't Ignore It} page}.

\subsection*{Religious Accommodations}

Campus policy requires faculty to provide reasonable accommodations for
students who, because of religious obligations, have conflicts with
scheduled exams, assignments or required attendance. Please communicate
the need for a religious accommodation in a timely manner, i.e. at least a week before the event. 

See the
\href{http://www.colorado.edu/policies/observance-religious-holidays-and-absences-classes-andor-exams}{{campus
policy regarding religious observances}} for full details.

\subsection*{Mental Health and Wellness}

The University of Colorado Boulder is committed to the well-being of all
students. If you~are struggling with personal stressors, mental health
or substance use concerns that are impacting academic or daily life,
please contact~\href{https://www.colorado.edu/counseling/}{{Counseling
and Psychiatric Services (CAPS)}} located in C4C or call (303) 492-2277,
24/7.~\\
~\\
Free and unlimited telehealth is also available
through~\href{https://www.colorado.edu/health/academiclivecare}{{Academic
Live Care}}.
The~\href{https://www.colorado.edu/health/academiclivecare}{Academic
Live Care} site also provides information about additional wellness
services on campus that are available to students.

}
\end{document}
