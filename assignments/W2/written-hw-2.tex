\documentclass{article}

\usepackage{fullpage,amsmath,amsthm,graphicx,enumitem}
\usepackage{hyperref}
\usepackage{amssymb}
\usepackage{wasysym}
\usepackage{ifthen}
\usepackage[symbol]{footmisc}
\newboolean{solutions}

\theoremstyle{definition}
\newtheorem{question}{Question}

\newcommand{\option}{{\Large$\Square$ }}

\title{ASEN 3728 Aircraft Dynamics\\Written Homework 2}

\date{Due date listed on Gradescope.}

\begin{document}

%%%% UPDATE TO RENDER SOLUTIONS %%%%
\setboolean{solutions}{false} 

\maketitle

\begin{question}
    Consider the following simplified quadrotor translational dynamics equation,
    \begin{equation*}
        m\dot{u}_E = Z_c\sin\theta-\nu |u_E| u_E \text{,}
    \end{equation*}
    where $u_E$ is the velocity in the \emph{inertial} x axis\footnote{$u_E$ should not be confused with $u^E$, the earth-relative velocity component in the \emph{body} x axis.}, and the control force $Z_c$ is managed by an automatic control system to maintain zero vertical acceleration, that is $Z_c=-\frac{mg}{\cos\theta}$. This equation can be used to solve for a steady forward-flight trim condition characterized by $\dot{u}_E = 0$. Assume $\psi=\phi=0$ and that there is no wind.

\begin{enumerate}[label=\alph*),nosep]
    \item Draw a diagram of the quadrotor in forward flight from the side (i.e. the y axis is pointing directly toward you). Label the angle $\theta$, the direction of positive $u_E$, and the aerodynamic, control, and gravity force vectors, $^a\mathbf{f}$, $^c\mathbf{f}$, and $^g\mathbf{f}$ assuming $u_E$ is positive.\footnote{You can label $-\theta$ instead of $\theta$ if you would like to.}
    \item Write an equation for the pitch angle $\theta_0$ as a function of the forward flight velocity $u_{E,0}$, in a forward flight trim state.
    \item Linearize these dynamics about the trim state, that is, find a linear equation for $\Delta \dot{u}_E$ in terms of $\Delta u_E$ and $\Delta \theta$ assuming that $u_E$ is positive. Simplify so that every term in the equation has a disturbance variable (one that begins with $\Delta$) in it.\footnote{Hint: Use a Taylor series expansion to linearize $\tan\theta$ about $\theta_0$.}
\end{enumerate}

\end{question}
\vspace{0.1cm}
\ifthenelse{\boolean{solutions}}{\input{solutions/Ben_soln}}{}
\clearpage

\begin{question}
    A 10 kg quadrotor has state vector $\vec{x}$ = [0 m, 0 m, 0 m, 0 rad, $-\pi$/6 rad, 0 rad, 11 m/s, 4 m/s, 1 m/s, -0.1 rad/s, 0.2 rad/s, 0.4 rad/s]. There is wind with velocity $\vec{W}_B^E = [2,2,0]^T$ m/s. Each rotor applies a control force of magnitude 30 N to the aircraft, and $\nu$ = $10^{-2}$ kg/m. Considering forces due to gravity, aerodynamic drag, and the rotor control, what is $\dot{\vec{V}}_B^E$?
\end{question}
\vspace{0.1cm}
\ifthenelse{\boolean{solutions}}{\input{solutions/Scott_soln}}{}

\clearpage

\begin{question}
    Consider a aircraft flying near a mountain ridge as shown in the diagram. The aircraft's pitot tube is used to measure the air-relative velocity in body coordinates:
     \[ \vec{V}^W_B = [100 \, m/s, 0, 0]^T \]
    The background wind is known to be blowing due East from a recent weather report at 30 m/s. The aircraft is flying at an altitude $z_E = -3000$ m, with a heading angle $\psi = -60$\textdegree{}. The ridge is located located 20 km West of the current position and has a height of 4000 m. 

     % \[ \vec{W}^E = [0 , 30 \, m/s, 0]^T \] 
    
    % \begin{figure}[htbp]
    %     \centering
    %     \includegraphics[width=1\textwidth]{AC Diagram.JPG}
    % \end{figure}
    
\begin{enumerate}[label=(\alph*)]
    \item How long will it take for the aircraft to reach the ridge?
    \item If the pilot wants to clear the ridge, how fast must the aircraft climb? Assume no change in $W$ or the internal x and y components of $\vec{V}^E$.
\end{enumerate}
\end{question}
\vspace{0.1cm}
\ifthenelse{\boolean{solutions}}{\input{solutions/Yusif_soln}}{}

\end{document}
