\begin{question}
    While flying near sea-level at $u_0 = 10$ m/s with $\theta_0 = 0$, the TTwistor aircraft from homework P3 has the lateral dynamics matrix shown below. For all parts, show work and/or describe any code used.
    \begin{align*}
        \mathbf{A}_{lat} = 
        \begin{bmatrix}
         -0.2472 & -0.0671 & -9.7797 & 9.8100 \\
         -0.7966 & -16.5375 & 1.8114 & 0 \\
         0.4607 & -0.3451 & -0.4586 &  0 \\
         0 & 1.0000 & 0 & 0
        \end{bmatrix}
    \end{align*}
    \begin{enumerate}
        \item Find the time constant of the roll mode, time constant of the spiral mode, and natural frequency and damping ratio of the Dutch roll mode of the TTwistor using the full lateral matrix.
        \item Calculate the time constant of the roll mode using the roll mode approximation. Compare this to the time constant found in part 1 (using \% error). Is this a good approximation?
        \item Calculate the natural frequency and damping ratio of the Dutch roll mode using the Dutch roll approximation. Compare this to the natural frequency and damping ratio found in part 1 (using \% error).Is this a good approximation?
        % \item Calculate the time constant of the spiral mode using the ``2x2" spiral mode approximation and the characteristic equation spiral mode approximation. Compare these to the time constant of the spiral mode found in part 1 (using \% error). Is this a good approximation?
        \item Calculate the time constant of the roll and spiral modes using the roll and spiral mode approximation. Compare this to the time constants found in part 1 (using \% error). Is this a good approximation?
    \end{enumerate}

\end{question}