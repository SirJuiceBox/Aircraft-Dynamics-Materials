\documentclass{article}

\usepackage{fullpage,amsmath,amsthm,graphicx,enumitem}
\usepackage{hyperref}
\usepackage{amssymb}
\usepackage{wasysym}

\theoremstyle{definition}
\newtheorem{question}{Question}

\newcommand{\option}{{\Large$\Square$ }}

\title{ASEN 3728 Aircraft Dynamics\\Written Homework 1}

\date{Due date listed on Gradescope.}

\begin{document}

\maketitle

\begin{question}(From Exam 1, Fall 2020)
    You are given: i.) an aircraft’s inertial velocity in body coordinates, ii.) the aircraft’s air-relative velocity in the wind frame coordinates, and iii.) the velocity of the wind relative to the inertial frame in inertial coordinates. Further, you are given iv.) the rotation matrix from the wind frame to the body frame, and v.) the rotation matrix from the body frame to the inertial frame. State the correct vector or matrix notation for each term, and then write a single equation that relates them all.
\end{question}

\vspace{6cm}

\begin{question}(From Exam 1, 2015)
    TRUE or FALSE. Two airshow performers fly maneuvers starting from the same orientation. The first performer changes attitude by 90 degrees around the body y axis, 90 degree about the body x axis, then 90 degrees in the body z axis, in that order. The second changes attitude by 90 degrees around the body x axis, 90 degree about the body z axis, then 90 degrees in the body y axis, in that order. The second aircraft now has the same attitude as the first.  Justify your answer.
\end{question}
\vspace{1ex}
\option TRUE \hspace{1cm} \option FALSE

\clearpage

\begin{question}(From Exam 1, Fall 2022)
    TRUE or FALSE. If an aircraft is flying with ${\bf v}^E_B = [15, 0, 2]^T$ m/s, $\beta = 6^{\circ}$, and $\psi = -6^{\circ}$ then the inertial wind velocity ${\bf w}^E$ cannot be zero.  Justify your answer.
\end{question}
\vspace{1ex}
\option TRUE \hspace{1cm} \option FALSE


\vspace{8cm}

\begin{question}
    The inertial velocity in body coordinates of an aircraft is $[18; 0; -5]$ m/s. The orientation of the aircraft is given by Euler angles $\phi = 9^{\circ}$, $\theta = -2^{\circ}$, and $\psi = 33^{\circ}$. What is $\mathbf{v}_B^E$? What is $\mathbf{v}_E^E$?
\end{question}

\end{document}
