\begin{question}

Consider a quadrotor with $I_y=3\ \text{kg m}^2$. Perform a closed-loop modal analysis of the quadrotor's longitudinal $\Delta \theta$ and $\Delta q$ motion using the information below. Subscripts on the natural frequencies and damping ratios correspond to the eigenvalues of the state space model for the motion.
\begin{align*}
    \Delta M_c &= -k_1 \Delta q - k_2 \Delta \theta \\
    \omega_{n,1,2} &= 1.8 \quad \text{[rad/s]} \\
    \zeta_{1,2} &= 0.5
\end{align*}

\begin{enumerate}
    \item Calculate the values of $k_1$ and $k_2$, as well as the eigenvalues $\lambda_1$ and $\lambda_2$ of the size $2\times2$ state space model $\mathbf{A}$ matrix.
    \item At $t=0$, the quadrotor's state is $\mathbf{x}(0) = 5\mathbf{v}_1 + \mathbf{v}_2$. The vectors $\mathbf{v}_1$ and $\mathbf{v}_2$ are the unit-length eigenvectors of the $\mathbf{A}$ matrix which correspond to the eigenvalues $\lambda_1$ and $\lambda_2$. Write the solution $\mathbf{x}(t)=(\Delta \theta(t),\Delta q(t))^T$ in terms of $t$, $\lambda_1$, $\lambda_2$, $\mathbf{v}_1$, and $\mathbf{v}_2$. 
    \item Calculate the eigenvectors $\mathbf{v}_1$ and $\mathbf{v}_2$.
    \item Describe the behavior of $\Delta \theta$ over time using the eigenvalues you calculated in Part (1).
    % \item If you were given the values of $\lambda_1$ and $\mathbf{v}_1$ and both were complex, would you be able to write down $\lambda_2$ and $\mathbf{v}_2$ if you didn't know $k_1$ or $k_2$? Why or why not? 
\end{enumerate}
\end{question}