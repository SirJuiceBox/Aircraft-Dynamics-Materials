\begin{question}
    Consider the longitudinal dynamics of the linearized quadrotor EOM:
    \begin{equation*}
        \left(\begin{array}{c}
        \Delta \dot{x}_E \\
        \Delta \dot{u} \\
        \Delta \dot{\theta} \\
        \Delta \dot{q}
        \end{array}\right)=\left(\begin{array}{c}
        \Delta u \\
        -g \Delta \theta \\
        \Delta q \\
        \frac{1}{I_u} \Delta M_c
        \end{array}\right)
    \end{equation*}
    where $\Delta M_c$ is defined in terms of $k_1$ and $k_2$ as in question 1. 

    \begin{enumerate}
        \item Suppose a closed-loop modal analysis of the system was performed and you were given only the following values:
    \begin{equation*}
        \lambda_1 = -1.5+4.2i \hspace{10pt} \lambda_2 = -0.0023+0.037i
    \end{equation*}
    \begin{equation*}
        \mathbf{v}_1 =  \left(\begin{array}{c}
        0.005+0.0021i \\
        0.075+0.0019i \\
        0.0085+0.0065i \\
        0.05 %Too challenging?
        \end{array}\right) 
        \hspace{10pt} \mathbf{v}_2 = \left(\begin{array}{c}
        0.006+0.0089i \\
        0.095+0.0030i \\
        0.0075+0.0015i \\
        0.0120+0.0030i %Too challenging?
        \end{array}\right) 
    \end{equation*}
    True or False: It is possible to determine $\lambda_3$, $\lambda_4$, $\mathbf{v}_3$, and $\mathbf{v}_4$ if $k_1$ or $k_2$ are unknown. Explain your answer.
    
    \vspace{1ex}
    \option TRUE \hspace{1cm} \option FALSE
    \vspace{1ex}
    
    \item Consider the following eigenvalues for another quadrotor system with the same dynamics described above:
    \begin{equation*}
        \lambda_1 = -1.5 \hspace{10pt} \lambda_2 = -0.0023+0.037i
    \end{equation*}
    True or False: It is possible to determine $\lambda_3$, $\lambda_4$ if $k_1$ or $k_2$ are unknown. Explain your answer.
        
    \vspace{1ex}
    \option TRUE \hspace{1cm} \option FALSE 
    \end{enumerate}
\end{question}
